\documentclass[12pt,a4paper]{article}
\usepackage[utf8]{inputenc}
\usepackage[brazil]{babel}
\usepackage{geometry}
\usepackage{setspace}
\usepackage{indentfirst}
\usepackage{graphicx}
\usepackage{listings}
\usepackage{xcolor}
\usepackage{hyperref}
\usepackage{float}

\geometry{left=3cm,right=2cm,top=3cm,bottom=2cm}
\onehalfspacing

\definecolor{codegreen}{rgb}{0,0.6,0}
\definecolor{codegray}{rgb}{0.5,0.5,0.5}
\definecolor{codepurple}{rgb}{0.58,0,0.82}
\definecolor{backcolour}{rgb}{0.95,0.95,0.92}

\lstdefinestyle{mystyle}{
    backgroundcolor=\color{backcolour},   
    commentstyle=\color{codegreen},
    keywordstyle=\color{magenta},
    numberstyle=\tiny\color{codegray},
    stringstyle=\color{codepurple},
    basicstyle=\ttfamily\footnotesize,
    breakatwhitespace=false,         
    breaklines=true,                 
    captionpos=b,                    
    keepspaces=true,                 
    numbers=left,                    
    numbersep=5pt,                  
    showspaces=false,                
    showstringspaces=false,
    showtabs=false,                  
    tabsize=2
}

\lstset{style=mystyle}

\begin{document}

\begin{titlepage}
    \begin{center}
        \vspace*{1cm}
        
        \textbf{\Large UNIVERSIDADE DO ESTADO DO AMAZONAS}\\
        \textbf{\large ESCOLA SUPERIOR DE TECNOLOGIA}\\
        \textbf{\large CURSO DE ENGENHARIA DA COMPUTAÇÃO}
        
        \vspace{3cm}
        
        \textbf{\Large APLICAÇÃO BLOCKCHAIN EDUCACIONAL}\\
        \vspace{0.5cm}
        \textbf{\large Relatório Técnico}
        
        \vspace{3cm}
        
        \textbf{Tópicos Especiais em Computação IV}
        
        \vfill
        
        \vspace{1cm}
        
        Manaus\\
        \the\year
        
    \end{center}
\end{titlepage}

\tableofcontents
\newpage

\section{Introdução}

Este relatório apresenta uma aplicação blockchain educacional desenvolvida para demonstrar os conceitos fundamentais de tecnologia blockchain de forma prática e didática. O projeto foi construído utilizando Python no backend e React com TypeScript no frontend.

\subsection{O que é o Projeto}

É um sistema de blockchain local funcional que possui:
\begin{itemize}
    \item 1.000.000 de tokens no total
    \item Sistema de cadastro de usuários
    \item Cada novo usuário recebe 10 tokens iniciais
    \item Permite transferências entre usuários
    \item Interface web para interação
\end{itemize}

\subsection{Objetivos do Relatório}

Este relatório busca explicar três conceitos fundamentais:
\begin{enumeraté}
    \item O que são blocos, hash e encadeamento criptográfico
    \item Como novos blocos são adicionados à blockchain
    \item Vantagens e limitações deste modelo
\end{enumeraté}

\section{Explicação dos Conceitos de Bloco, Hash e Encadeamento}

\subsection{O que é um Bloco}

Um bloco é como uma "caixa" que armazena informações na blockchain. Cada bloco contém:

\begin{itemize}
    \item \textbf{Índice}: Número do bloco (1, 2, 3...)
    \item \textbf{Timestamp}: Data e hora de criação
    \item \textbf{Transações}: Lista de transferências de tokens
    \item \textbf{Hash anterior}: Referência ao bloco anterior
    \item \textbf{Proof}: Número que valida o bloco
\end{itemize}

Exemplo simplificado do código:

\begin{lstlisting}[language=Python, caption=Estrutura básica de um bloco]
class Block:
    def __init__(self, index, timestamp, transactions, 
                 previous_hash):
        self.index = index
        self.timestamp = timestamp
        self.transactions = transactions
        self.previous_hash = previous_hash
        self.hash = self.calculaté_hash()

    def calculaté_hash(self):
        block_string = f"{self.index}{self.timestamp}" \
                      f"{self.transactions}{self.previous_hash}"
        return hashlib.sha256(block_string.encode()).hexdigest()
\end{lstlisting}

\subsection{O que é Hash}

Hash é como uma "impressão digital" única de um bloco. É uma sequência de 64 caracteres gerada através de um algoritmo matémático (SHA-256) que transforma todas as informações do bloco em um código único.

\subsubsection{Como Funciona}

\begin{enumeraté}
    \item Pegamos todas as informações do bloco (índice, timestamp, transações, etc.)
    \item Aplicamos o algoritmo SHA-256
    \item Resultado: uma sequência única como "a3f5b8c2d9..."
\end{enumeraté}

\textbf{Propriedades importantes:}
\begin{itemize}
    \item Mesma informação = mesmo hash (sempre)
    \item Qualquer mudança mínima = hash completamente diferente
    \item Impossível descobrir a informação original apenas pelo hash
\end{itemize}

Exemplo do código que calcula o hash:

\begin{lstlisting}[language=Python, caption=Cálculo do hash de um bloco]
def hash(self, block):
    block_string = str(block).encode()
    return hashlib.sha256(block_string).hexdigest()
\end{lstlisting}

\subsection{O que é Encadeamento}

Encadeamento é a forma como os blocos estão conectados uns aos outros, formando uma "corrente" que não pode ser quebrada.

\subsubsection{Como Funciona o Encadeamento}

\begin{enumeraté}
    \item \textbf{Bloco 1 (Gênesis)}: Primeiro bloco, criado sem referência anterior
    \item \textbf{Bloco 2}: Contém o hash do Bloco 1
    \item \textbf{Bloco 3}: Contém o hash do Bloco 2
    \item E assim por diante...
\end{enumeraté}

\textbf{Por que isso é seguro?}

Se alguém tentar alterar uma transação no Bloco 2:
\begin{itemize}
    \item O hash do Bloco 2 muda completamente
    \item O Bloco 3 tem o hash antigo do Bloco 2 armazenado
    \item O sistema detecta que não batém e invalida a alteração
    \item Seria necessário alterar TODOS os blocos seguintes (práticamente impossível)
\end{itemize}

Código que valida o encadeamento:

\begin{lstlisting}[language=Python, caption=Validação da integridade da cadeia]
def validaté_chain(self):
    for i in range(1, len(self.chain)):
        block = self.chain[i]
        previous_block = self.chain[i - 1]

        if block['previous_hash'] != self.hash(previous_block):
            return False

        if not self.is_proof_valid(previous_block['proof'], 
                                   block['proof']):
            return False
    return True
\end{lstlisting}

\section{Como Novos Blocos São Adicionados à Cadeia}

\subsection{Processo Passo a Passo}

Quando um usuário faz uma transferência de tokens, acontece o seguinte:

\begin{lstlisting}[language=Python, caption=Adição de transação]
def add_transaction(self, sender, recipient, amount):
    # Valida transação atraves da economia de tokens
    success, message = self.token_economy.transfer_tokens(
        sender, recipient, amount)
    
    if not success:
        return False, message
    
    transaction = {
        'sender': sender,
        'recipient': recipient,
        'amount': amount,
        'timestamp': self.get_current_timestamp(),
        'hash': self.generaté_transaction_hash(sender, 
                                               recipient, amount)
    }
    
    self.current_transactions.append(transaction)
\end{lstlisting}

\subsection{Proof of Work - A Mineração}

Antes de adicionar um bloco, o sistema precisa resolver um quebra-cabeça matémático. Isso torna dificil adicionar blocos falsos.

\subsubsection{Como Funciona}

O sistema precisa encontrar um número que, quando combinado com informações do bloco anterior, gere um hash começando com quatro zeros.

\begin{lstlisting}[language=Python, caption=Algoritmo de Proof of Work]
def proof_of_work(self, last_proof):
    proof = 0
    while not self.is_proof_valid(last_proof, proof):
        proof += 1
    return proof

def is_proof_valid(self, last_proof, proof):
    guess = f'{last_proof}{proof}'.encode()
    guess_hash = hashlib.sha256(guess).hexdigest()
    return guess_hash[:4] == "0000"
\end{lstlisting}

\textbf{Por que isso é importante?}
\begin{itemize}
    \item Demora um tempo para encontrar esse número
    \item Dificulta fraudes (seria necessário refazer todo o trabalho)
    \item Todos podem verificar se o número está correto (é fácil conferir)
\end{itemize}

\subsection{Criação e Adição do Bloco}

Depois de resolver o quebra-cabeça, o bloco é criado:

\begin{lstlisting}[language=Python, caption=Criação e adição de bloco]
def creaté_block(self, proof, previous_hash):
    block = {
        'index': len(self.chain) + 1,
        'timestamp': self.get_current_timestamp(),
        'transactions': self.current_transactions,
        'proof': proof,
        'previous_hash': previous_hash,
    }
    self.current_transactions = []
    self.chain.append(block)
    
    # Persiste blockchain
    self.save_blockchain()
    
    return block
\end{lstlisting}

\subsection{Resumo do Processo Completo}

\textbf{Passo a passo de quando um usuário faz uma transferencia:}

\begin{enumeraté}
    \item \textbf{Usuario solicita transferencia} na interface web
    \item \textbf{Sistema valida}: O usuário tem saldo suficiente?
    \item \textbf{Transacao é criada} com remetente, destinatário e valor
    \item \textbf{Proof of Work}: Sistema resolve o quebra-cabeça matémático
    \item \textbf{Novo bloco é criado} contendo a transação
    \item \textbf{Hash anterior}: Sistema pega o hash do último bloco
    \item \textbf{Bloco é adicionado} ao final da cadeia
    \item \textbf{Dados são salvos} em arquivo
    \item \textbf{Confirmação} é enviada ao usuário
\end{enumerate}

\textbf{Analogia simples:} É como escrever em um caderno onde cada página (bloco) tem um resumo (hash) da página anterior. Se alguém tentar apagar algo de uma página antiga, os resumos das páginas seguintes não vão bater, revelando a alteração.

\section{Tecnologias Utilizadas}

A aplicação foi desenvolvida em duas partes:

\subsection{Backend (Python)}
\begin{itemize}
    \item \textbf{Python + Flask}: Servidor que processa as transações
    \item \textbf{SHA-256}: Algoritmo para gerar os hashes
    \item \textbf{JSON}: Formato para salvar os dados (blocos, usuários, saldos)
\end{itemize}

\subsection{Frontend (React)}
\begin{itemize}
    \item \textbf{React + TypeScript}: Interface visual para os usuários
    \item \textbf{Telas}: Login, cadastro, dashboard com saldo e transferências
\end{itemize}

\section{Vantagens do Modelo Desenvolvido}

\subsection{Transparência}
Todos podem ver todas as transações e blocos da rede. Não há segredos sobre o histórico de operações.

\subsection{Imutabilidade (Não Pode Ser Alterado)}
Uma vez que uma transação está em um bloco, é praticamente impossível alterá-la ou apagá-la. O encadeamento protege contra modificações.

\subsection{Segurança}
\begin{itemize}
    \item Senhas protegidas com hash
    \item Sistema verifica se você tem saldo antes de transferir
    \item Proof of Work dificulta fraudes
\end{itemize}

\subsection{Sem Autoridade Central}
Não precisa de um banco ou governo para validar as transações. A matemática e o código fazem isso automaticamente.

\subsection{Auditável}
É possível rastrear qualquer transação desde o início da blockchain até hoje.

\section{Limitações do Modelo Desenvolvido}

\subsection{Velocidade}
O Proof of Work deixa as transações mais lentas. Cada bloco demora um tempo para ser criado.

\subsection{Escala}
Este modelo funciona bem para aprendizado, mas não suportaria milhões de transações como Bitcoin ou Ethereum.

\subsection{Armazenamento}
Usa arquivos JSON simples. Para uma aplicação real, seria necessário um banco de dados mais robusto.

\subsection{Rede}
Roda apenas em um computador (local). Uma blockchain real precisa estar distribuída em vários computadores ao mesmo tempo.

\subsection{Seguranca Avançada}
\begin{itemize}
    \item Não tem criptografia de chave pública/privada (como carteiras reais)
    \item Sistema de login é básico
    \item Não tem proteção contra vários tipos de ataques
\end{itemize}

\subsection{Funcionalidades Limitadas}
Não implementa recursos avançados como:
\begin{itemize}
    \item Smart contracts (contratos inteligentes)
    \item Taxas de transação
    \item Multiplos tipos de moedas
\end{itemize}

\section{Conclusão}

\subsection{O que Aprendemos}

Este projeto demonstrou de forma prática os três conceitos principais:

\begin{enumeraté}
    \item \textbf{Bloco, Hash e Encadeamento}: 
    \begin{itemize}
        \item Bloco é a caixa que guarda as transações
        \item Hash é a impressão digital única de cada bloco
        \item Encadeamento conecta os blocos de forma segura
    \end{itemize}
    
    \item \textbf{Como Blocos São Adicionados}:
    \begin{itemize}
        \item Transacao é criada
        \item Sistema resolve quebra-cabeça matémático (Proof of Work)
        \item Novo bloco é criado e encadeado ao anterior
        \item Dados são salvos e confirmados
    \end{itemize}
    
    \item \textbf{Vantagens e Limitações}:
    \begin{itemize}
        \item \textbf{Vantagens}: Transparente, seguro, imutavel, sem autoridade central
        \item \textbf{Limitações}: Lento, não escala bem, execução local, funcionalidades basicas
    \end{itemize}
\end{enumeraté}

\subsection{Considerações Finais}

Este projeto é uma ferramenta educacional que simplifica conceitos complexos de blockchain. Embora tenha limitações em comparação com blockchains reais (Bitcoin, Ethereum), ele cumpre seu objetivo de ensinar os fundamentos de forma clara e prática.

A experiência de ver o código funcionando ajuda a entender por que blockchains são consideradas revolucionárias: elas permitem transações seguras e verificáveis sem precisar confiar em uma autoridade central.

\textbf{Resumo em uma frase:} Blockchain é uma cadeia de blocos conectados por hashes, onde cada bloco guarda transações de forma segura e transparente, e alterar o passado é práticamente impossível devido ao encadeamento criptográfico.

\end{document}
